%\documentclass[13pt,aspectratio=169]{beamer}% Used to establish the beamer environment
%
\documentclass[20pt]{beamer} % scale up the font largest available is 20
\geometry{papersize={16in,9in}} % make much larger. needs testing
%
%
\usepackage{tikz} % Used for vectorized drawings
\usetikzlibrary{shapes.geometric}
\usetikzlibrary{math}
%
\usepackage{amsmath} % Math typesetting package
\usepackage{amssymb} % Math symbols
\usepackage{float}  % Defines floating objects (Tables, Figures, etc...)
\usepackage{pgfplots} % Package for plotting
\usepackage{multirow}
\usepackage{color,soul} % 
\usepackage{mathtools}
\usepackage{appendix}
\usepackage{siunitx}
\usepackage[normalem]{ulem}
\usepackage{upgreek}
\usepackage[numbers]{natbib}
\usepackage{subfigure}
% Theme choice (black and white, essentially)

\usecolortheme{dove}

% Frame titles
\setbeamerfont{frametitle}{size=\Large,series=\bfseries}
% Block title settings
\setbeamerfont{block title}{size=\large,series=\bfseries}
% Footer with acme lab, logo and frame number 
\addtobeamertemplate{footline}{%
  \begin{tikzpicture}
    % Crimson PMS 201
    \definecolor{mycolor}{RGB}{158,27,50}
    %
    \filldraw[color=mycolor, fill=mycolor, very thick] (-2,-1) rectangle (\textwidth,.27);
    
    \node[right, anchor= south west, text =white] (logo) at  (-2,-1) {\includegraphics[width= 1cm]{figures/A-Square-Logo-4c_Official.jpg}};
    
    \node[right, anchor=west, text =white] at (logo.east) {\normalsize ACME Lab};
    
   \node [left,anchor=south east, text =white] at (.93\textwidth,-.67) {\normalsize \insertframenumber{}};
\end{tikzpicture}}

% Removes navigation symbols
\setbeamertemplate{navigation symbols}{}

% Title and author
\title{Grain and element precipitate distribution}
\author{Ezra Mengiste, Matthew Kasemer}
\setbeamerfont{title}{size=\Huge}


\graphicspath{{/home/etmengiste/jobs/SERDP/dense_mesh/layer_1/imgs/}} % Path to graphics

%%%%% Begin document
\begin{document}

\newcommand{\chilCol}{grey}
% Make title page
% Include body slides here
\maketitle

\newcommand{\makeDataslide}[2]{
\begin{frame}{SERDP: RCL = #1.#2}

    \begin{columns}
        \begin{column}{0.1\textwidth}
        \begin{block}{\textbf{$\tau_{crss}$}}
        \includegraphics[width=\textwidth,centering]{mesh_rcl0_49_step0-scale3d.png}            
        \end{block}
            
        \end{column}
        \begin{column}{0.3\textwidth}
        \begin{block}{Step 0}
            \includegraphics[width=\textwidth,trim={0 0 0 0},clip]{mesh_rcl#1_#2_step0.png}                  
        \end{block}
        \end{column}
        %
        \begin{column}{0.3\textwidth}
        \begin{block}{Step 16 $\varepsilon=0.26$}
            \includegraphics[width=\textwidth,trim={0 0 0 0},clip]{mesh_rcl#1_#2_step16.png}                       
        \end{block}
        \end{column}
        \begin{column}{0.3\textwidth}
            \includegraphics[width=\textwidth,trim={0 0 0 0},clip]{mesh_rcl#1_#2_svs.png}                   
        \end{column}
    \end{columns}
\end{frame}
}
\makeDataslide{1}{0}
\makeDataslide{0}{915}
\makeDataslide{0}{83}
\makeDataslide{0}{745}
\makeDataslide{0}{66}
\makeDataslide{0}{575}
\makeDataslide{0}{49}
\makeDataslide{0}{405}
\makeDataslide{0}{32}
\makeDataslide{0}{235}
%\begin{frame}
    %
    \Large
    This slide is an example of having two columns, one that is 30\% of the slide width, the other that is 40\% of the slide width.
    %
    \begin{columns}
        \begin{column}{0.3\textwidth}
            \begin{enumerate}
                \item One side here
            \end{enumerate}
        \end{column}
        %
        \begin{column}{0.4\textwidth}
        \end{column}
        %
        \begin{column}{0.4\textwidth}
        \end{column}
    \end{columns}
    %
\end{frame}
%%%%% End document
\end{document}
